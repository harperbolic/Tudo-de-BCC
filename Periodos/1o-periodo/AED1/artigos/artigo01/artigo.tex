%%%%%%%%%%%%%%%%%%%%%%%%%%%%%%%%%%%%%%%%%%%%%%%%%%%%%%%%%%%%%%%%%%%%%%%%%%%%%%%%%%%%%%%%%%%%%%%%%%%%%%%
%%%%%%%%%%%%%% Template de Artigo Adaptado para Trabalho de Diplomação do ICEI %%%%%%%%%%%%%%%%%%%%%%%%
%% codificação UTF-8 - Abntex - Latex -  							     %%
%% Autor:    Fábio Leandro Rodrigues Cordeiro  (fabioleandro@pucminas.br)                            %% 
%% Co-autor: Prof. João Paulo Domingos Silva, Harison da Silva e Anderson Carvalho                   %%
%% Revisores normas NBR (Padrão PUC Minas): Helenice Rego Cunha e Prof. Theldo Cruz                  %%
%% Versão: 1.1     18 de dezembro 2015                     	                                     %%
%%%%%%%%%%%%%%%%%%%%%%%%%%%%%%%%%%%%%%%%%%%%%%%%%%%%%%%%%%%%%%%%%%%%%%%%%%%%%%%%%%%%%%%%%%%%%%%%%%%%%%%


\documentclass[a4paper,12pt,Times]{article}
\usepackage{abakos}  %pacote com padrão da Abakos baseado no padrão da PUC

%%%%%%%%%%%%%%%%%%%%%%%%%%%
%Capa da revista
%%%%%%%%%%%%%%%%%%%%%%%%%%

%\setcounter{page}{80} %iniciar contador de pagina de valor especificado
\newcommand{\monog}{}
\newcommand{\monogES}{}
\newcommand{\tipo}{Artigo }  % Especificar a seção tipo do trabalho: Artigo, Resumo, Tese, Dociê etc
\newcommand{\origem}{Brasil }
\newcommand{\editorial}{Belo Horizonte, p. 01-11, mar. 2024}  % p. xx-xx – páginas inicial-final do artigo
\newcommand{\lcc}{\scriptsize{Licença Creative Commons Attribution-NonCommercial-NoDerivs 3.0 Unported}}

%%%%%%%%%%%%%%%%%INFORMAÇÕES SOBRE AUTOR PRINCIPAL %%%%%%%%%%%%%%%%%%%%%%%%%%%%%%%
\newcommand{\AutorA}{Harper Moreira Mascarenhas}
\newcommand{\funcaoA}{}
\newcommand{\emailA}{harper.mascarenhas@sga.pucminas.br}
\newcommand{\cursA}{Aluna de Graduação em Ciência da Computação}

\newcommand{\AutorB}{Fábio Leandro Rodrigues Cordeiro}
\newcommand{\funcaoB}{}
\newcommand{\emailB}{fabioleandro@pucminas.br}
\newcommand{\cursB}{Professor do Programa de Graduação em Ciência da Computação}
% 
% Definir macros para o nome da Instituição, da Faculdade, etc.
\newcommand{\univ}{Pontifícia Universidade Católica de Minas Gerais}

\newcommand{\keyword}[1]{\textsf{#1}}

\begin{document}
% %%%%%%%%%%%%%%%%%%%%%%%%%%%%%%%%%%
% %% Pagina de titulo
% %%%%%%%%%%%%%%%%%%%%%%%%%%%%%%%%%%

\begin{center}
\includegraphics[scale=0.2]{figuras/brasao.jpg} \\
PONTIFÍCIA UNIVERSIDADE CATÓLICA DE MINAS GERAIS \\
Instituto de Ciências Exatas e de Informática

% \vspace{1.0cm}

\end{center}

 \vspace{0cm} {
 \singlespacing \Large{\monog \symbolfootnote[1]{Artigo apresentado ao Instituto de Ciências Exatas e Informática da Pontifícia Universidade Católica de Minas Gerais como pré-requisito para obtenção do título de Bacharel em Ciência da Computação.} \\ }
  \normalsize{\monogES}
 }

\vspace{1.0cm}

\begin{flushright}
\singlespacing 
\normalsize{\AutorA \footnote{\funcaoA \cursA, \origem -- \emailA . }} \\
\normalsize{\AutorB \footnote{\funcaoB \cursB, \origem -- \emailB . }} \\
%\normalsize{\AutorC \footnote{\funcaoC \cursC, \origem -- \emailC . }} \\
%\normalsize{\AutorD \footnote{\funcaD \cursD, \origem -- \emailD . }} \\
%deixar com o valor `0` e usar o '*' no inicio da frase
% \symbolfootnote[0]{Artigo recebido em 10 de julho de 1983 e aprovado em 29 de maio 2012}
\end{flushright}
\thispagestyle{empty}

\vspace{1.0cm}

\begin{abstract}
\noindent
O resumo deverá conter pelo menos cento e cinquenta palavras de acordo com o padrão de normalização da ABNT.
Este artigo irá abordar as principais linguagens de programação voltadas a ambiente WEB usadas atualmente, 
comparando suas características de maneira a indicar o melhor uso para determinada linguagem. 
As linguagens serão divididas de acordo com 4 principais características: Interpretadas, compiladas, server-side e client-side.
O resumo deverá conter pelo menos cento e cinquenta palavras de acordo com o padrão de normalização da ABNT.
as linguagens serão divididas de acordo com 4 principais características: Interpretadas, compiladas, server-side e client-side.
O resumo deverá conter pelo menos cento e cinquenta palavras de acordo com o padrão de normalização da ABNT.
\\\textbf{\keyword{Palavras-chave: }} Template. \LaTeX. Abakos. Periódicos.
\end{abstract}

%%%%%%%%%%%%%%%%%%%%%%%%%%%%%%%%%%%%%%%%%%%%%%%%%%%%%%%%%
 \newpage    %%%% CASO QUEIRA QUE O RESUMO FIQUE EM UMA PAGINA E O ABSTRACT EM OUTRA
\selectlanguage{english}
\begin{abstract}
\noindent
The abstract should contain at least one hundred and fifty words in accordance with the standards of ABNT standard.
This present article will address the main features of the web programming languages, that are used currently, 
comparing its features as to indicate to indicate the better use of determined language. 
The linguage will be divided according with four major caracteristics: Interpreted, compiled, server-side and client-side.
This present article will address the main features of the web programming languages.
The abstract should contain at least one hundred and fifty words in accordance with the standards of ABNT standard.
The linguage will be divided according with four major caracteristics: Interpreted, compiled, server-side and client-side.
This present article will address the main features of the web programming languages.
The abstract should contain at least one hundred and fifty words in accordance with the standards of ABNT standard.
\\\textbf{\keyword{Keywords: }} Template. \LaTeX. Abakos. Periodics.
\end{abstract}

\selectlanguage{brazilian}
 \onehalfspace  % espaçamento 1.5 entre linhas
 \setlength{\parindent}{1.25cm}

%%%%%%%%%%%%%%%%%%%%%%%%%%%%%%%%%%%%%%%%%%%%%%%%%
%% INICIO DO TEXTO
%%%%%%%%%%%%%%%%%%%%%%%%%%%%%%%%%%%%%%%%%%%%%%%%%

\include{textos}

%%%%%%%%%%%%%%%%%%%%%%%%%%%%%%%%%%%
%% FIM DO TEXTO
%%%%%%%%%%%%%%%%%%%%%%%%%%%%%%%%%%%

% \selectlanguage{brazil}
%%%%%%%%%%%%%%%%%%%%%%%%%%%%%%%%%%%
%% Inicio bibliografia
%%%%%%%%%%%%%%%%%%%%%%%%%%%%%%%%%%%

 \newpage
\singlespace{
\renewcommand\refname{REFERÊNCIAS}
\bibliographystyle{abntex2-alf}
\bibliography{bibliografia}

}

\end{document}
